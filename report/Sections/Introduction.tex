\label{sec:Introduction}

In the domains of network science and graph theory, the identification of cohesive substructures within complex networks is a fundamental task with far-reaching implications. One such critical concept is the k-core of a graph, a mathematical construct that characterizes the inherent structure and resilience of a network. The process of identifying k-cores within large-scale networks is not a trivial one, and researchers have developed multiple algorithms to efficiently compute them \cite{9622845}. K-Cores decomposition can offer valuable insights into the organization of networks and have applications in various fields, including social networks, biology, transportation systems, among others.

The K-Cores Problem involves finding the maximal subgraph in which every vertex has a degree of at least \(k\). In other words, it aims to unveil clusters of nodes that are tightly interconnected, each having a minimum degree threshold \(k\). These subgraphs, known as k-cores, help us comprehend the intricate relationships and dependencies in various real-world networks. Identifying the k-cores can unveil tightly-knit communities of individuals in social networks, providing a better understanding of the underlying social structure \cite{dey2020network}.

In biological research, the K-Cores Problem finds applications in understanding protein-protein interaction networks \cite{Amin_2003}. And, when applied to the internet and web graphs, it sheds light on the robustness and connectivity patterns of websites \cite{Hamelin_2006}. Websites forming a k-core represent critical hubs in the web, indicating essential pages and content repositories.  Moreover, when applied to transportation networks, this decomposition can identify vital junctions or nodes in road systems or public transportation networks. Understanding these networks is crucial for urban planning, optimizing traffic flow, and ensuring the resilience of transportation infrastructures.

Through the identification and study of these cohesive cores, researchers and analysts can gain profound insights into various real-world systems, leading to better decision-making, enhanced design, and improved overall efficiency across diverse domains. K-Cores decomposition serves as a powerful analytical tool to dissect intricate networks into comprehensible substructures. 

With this report, we aim to present a comparative study of two distinct algorithms for finding k-cores within a given graph. Our primary objective is to explore and evaluate the strengths and weaknesses of these algorithms in terms of computational efficiency and scalability. By doing so, we aim to provide valuable insights into the selection of an appropriate algorithm based on specific network characteristics and research objectives. 

In the subsequent sections of this report, we provide an in-depth explanation of the chosen algorithms, present our experimental methodology, and report the findings of our comparative analysis.