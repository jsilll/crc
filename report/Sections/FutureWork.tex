The exploration of k-core decomposition algorithms has provided valuable insights into understanding the structure of complex networks. However, there are several avenues for future research that could enhance the field and contribute to the development of more efficient algorithms.

\subsection{Parallelization with OpenMP and MPI}

One of the key areas for future research involves investigating the parallelization of both k-core decomposition algorithms using Open Multi-Processing \footnote{\url{https://www.openmp.org/}} (OpenMP) and Message Passing Interface \footnote{\url{https://www.open-mpi.org/}} (MPI). Implementing parallel computing paradigms can significantly accelerate the execution of algorithms, especially when dealing with large-scale networks. OpenMP, a shared-memory parallelization framework, and MPI, a distributed-memory parallelization protocol, offer opportunities to exploit multi-core processors and distributed computing environments effectively. By parallelizing the algorithms, researchers can distribute the computational load across multiple processors or even across clusters of computers, thereby substantially reducing the processing time for k-core decomposition. This approach could be particularly beneficial for handling massive real-world networks encountered in various applications, such as social networks, biological networks, and communication networks.


\subsection{Exploring Performance Discrepancies}


Considering the impact of network characteristics on algorithm performance is crucial. Different types of networks, such as scale-free networks, small-world networks, or dense social networks, may pose unique challenges for k-core decomposition algorithms. Future research could focus on tailoring algorithms to specific network topologies, ensuring that the algorithms' performance remains consistent across various real-world scenarios. These efforts will not only advance the field of network analysis but also pave the way for more accurate and timely analyses of complex real-world networks.
