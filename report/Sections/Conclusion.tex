\label{Conclusion}

The K-Cores Problem represents a powerful paradigm in graph theory and network analysis, offering nuanced insights into the intricate world of interconnected systems. By delving into cohesive substructures within networks, this concept has far-reaching implications across various domains.

In social networks, the identification of K-Cores unravels tightly-knit communities, enhancing our understanding of human interactions and social dynamics. In biological networks, it aids in pinpointing essential elements, paving the way for advancements in disease research and drug discovery. When applied to internet and web graphs, K-Cores illuminate the backbone of the digital world, spotlighting pivotal websites and content hubs. Furthermore, in transportation and infrastructure networks, K-Cores pinpoint critical nodes, optimizing traffic flow and ensuring the robustness of urban planning strategies.

In order to identify the K-Cores in a graph, we evaluated two algorithms on a set of benchmarks, the Degree Pruning algorithm and a Depth-First Search approach to the K-Cores problem. Due to, primarily, the queue's size and the duplication of nodes, the Degree Pruning Algorithm is way less efficient than its Depth-First Search counter part. However, these conclusions are limited by the set of benchmarks used, since there could exist a set of benchmarks in which the conclusions are the opposite.

Although there is still some work to do in this area, as mentioned in section \ref{FutureWork}, this study contributed to identifying the most efficient algorithms for solving the K-Cores problem.

As networks continue to evolve and expand in complexity, the K-Cores Problem remains a cornerstone in understanding their underlying structures. Its applications in diverse fields underscore its significance, making it an indispensable asset in the toolkit of network analysts and researchers worldwide.
