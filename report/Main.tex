\documentclass[a4paper, 11pt]{article}

\usepackage{bm}  
\usepackage{float}
\usepackage{pdfsync}  
\usepackage{textcomp}
\usepackage{graphicx} 
\usepackage{fancyhdr}
\usepackage{memhfixc} 
\usepackage{etoolbox}
\usepackage{indentfirst}
\usepackage[utf8]{inputenc}
\usepackage{amsmath,amssymb}  
\usepackage[pdftex,bookmarks,colorlinks,breaklinks]{hyperref}  
\usepackage[top=3cm, bottom=3cm, left = 2cm, right = 2cm]{geometry} 
\usepackage[linesnumbered,ruled,vlined]{algorithm2e}

\AtBeginEnvironment{abstract}{\setlength{\parindent}{0pt}}

\geometry{a4paper} 
\hypersetup{linkcolor=black,citecolor=black,filecolor=black,urlcolor=black}

\title{Network Science Project \\ \textbf{Finding K-Cores of Large Graphs}}
\author{Guilherme Gonçalves \and João Silveira \and Manuel Brito}
\date{\today}

\begin{document}

\maketitle

\begin{abstract}
    In the fields of network analysis and graph theory, the K-Cores Problem is a potent paradigm that provides important insights into the complex webs of interrelated systems. Its ramifications cut across many fields, enabling analysts and researchers to solve the puzzles of intricate networks.
    
    The discovery of K-Cores in the context of social networks illuminates close-knit communities, improving our understanding of interpersonal relationships and social dynamics. It helps identify important components in biological networks, which leads to improvements in the study of diseases and the development of new drugs. K-Cores identify important websites and content hubs and reveal the core of the digital world when applied to the internet and web graphs. Furthermore, K-Cores identify crucial links in infrastructure and transportation networks, strengthening urban planning tactics and streamlining traffic.

    We compared two methods for K-Core identification in graphs: a Degree Pruning technique and a Depth-First Search strategy customised for the K-Cores problem. Due to considerations like queue size and node duplication, the results clearly favour the Depth-First Search strategy, making it far more efficient than the Degree Pruning Algorithm.

    As mentioned in section \ref{FutureWork}, further research in this area is necessary, but this work advances the continuing effort to find the best methods for resolving the K-Cores problem. The K-Cores Problem continues to be a key to comprehending the underlying architecture of networks as they develop and become more complicated. Its numerous uses in various industries highlight its importance and make it an essential tool for network analysts and researchers everywhere.
\end{abstract}


\clearpage
\tableofcontents
\clearpage


\section{Introduction}
\label{sec:Introduction}

In the realm of network science and graph theory, the identification of cohesive substructures within complex networks is a fundamental task with far-reaching implications. One such critical concept is the k-core of a graph, a mathematical construct that characterizes the inherent structure and resilience of a network. K-cores offer valuable insights into the core-periphery organization of networks and have applications in various fields, including social network analysis, biology, transportation systems, and more.

The k-core of a graph represents a maximal subgraph where every vertex is connected to at least k other vertices within the subgraph. This concept is essential for understanding the robustness and functionality of real-world networks. As the value of k increases, the resulting k-core becomes progressively denser, unveiling the inner layers of a network's connectivity.

The process of identifying k-cores is not a trivial one, and researchers have developed multiple algorithms to efficiently compute them. These algorithms serve as the cornerstone for uncovering the intricate hierarchical structure within networks and have significant implications for various network-related applications.

This report embarks on a comparative study of two distinct algorithms for finding k-cores within a given graph. Our primary objective is to explore and evaluate the strengths and weaknesses of these algorithms in terms of computational efficiency, accuracy, and scalability. By doing so, we aim to provide valuable insights into the selection of an appropriate algorithm based on specific network characteristics and research objectives.

The two algorithms under consideration, which will be thoroughly analyzed and compared in this report, are the \textit{Degree Pruning Algorithm} and an algorithm that uses \textit{DFS} to solve the problem. Both algorithms have gained recognition in the field of network science and have shown promise in identifying k-cores effectively. However, they employ different strategies and heuristics, which may lead to variations in their performance under different network scenarios.

In the subsequent sections of this report, we will delve into the theoretical foundations of k-cores, provide an in-depth explanation of the chosen algorithms, present our experimental methodology, and report the findings of our comparative analysis. By the conclusion of this study, readers will have a comprehensive understanding of the intricacies involved in identifying k-cores and will be equipped with insights to make informed decisions about the choice of algorithm for their specific network analysis tasks.


\section{Algorithms Overview}
\label{Algorithms}

\subsection{Degree Pruning Algorithm}

\subsubsection{Overview}

The degree-prunning algorithm is a widely used and efficient method for discovering k-cores within a graph. This algorithm progressively removes vertices and edges with degrees less than the desired k, ultimately revealing the k-core structure of the graph. It is especially effective in scenarios where the graph is sparse or moderately dense, making it a valuable tool for various network analysis tasks. 

Initially, the algorithm calculates the degree of every node and enqueues nodes with degrees less than $k$. Subsequently, it dequeues nodes from the queue one at a time, removes them from the graph, and decreases the degrees of their neighbors. If, at any stage, any of the neighbors' degrees fall below $k$, it is also placed in the queue for removal. Ultimately, only the k-core of the graph remains intact after this process. The code for this algorithm is presented in algorithm \ref{algo:degree-pruning}.

\begin{algorithm}[H]
    \label{algo:degree-pruning}
    \caption{Degree-Pruning Algorithm}
    \SetKwFunction{Kcores}{KCores}
    \SetKwProg{Fn}{Function}{:}{}
    \Fn{\Kcores{$g$, $k$}}{
        $q \leftarrow$ empty queue\;
        $degree \leftarrow$ array of size $|V|$\;
        \For{$i \leftarrow 0$ \KwTo $|V| - 1$}{
            $d \leftarrow$ degree of vertex $i$ in $g$\;
            \If{$d < k$}{
                $q$.enqueue($i$)\;
            }
            $degree[i] \leftarrow d$\;
        }
        \While{$q$ is not empty}{
            $u \leftarrow q$.dequeue()\;
            \ForEach{neighbor $v$ of $u$ in $g$}{
                \If{$degree[v] \geq k$}{
                    $degree[v] \leftarrow degree[v] - 1$\;
                    \If{$degree[v] < k$}{
                        $q$.enqueue($v$)\;
                    }
                }
            }
            Remove vertex $u$ from $g$\;
        }
        $cores \leftarrow$ empty list\;
        \For{$i \leftarrow 0$ \KwTo $|V| - 1$}{
            \If{degree of vertex $i$ in $g \geq k$}{
                $cores$.append($i$)\;
            }
        }
        \KwRet $cores$\;
    }
\end{algorithm}

\subsubsection{Complexity Analysis}

Considering that the algorithm permits the repeated insertion of a node up to its degree, the queue's size in the worst-case scenario will be the sum of the degrees of all nodes. It's established that the sum of the degrees of all nodes is limited by 2 times the number of edges ($2*E$). Consequently, the space required by the queue will be in the order of $O(E)$. Additionally, as the algorithm employs a vector of size $V$, the ultimate space complexity will be $O(V + E)$.

The time complexity will depend on two factors: the dequeuing loop (which has a time complexity of $O(E)$, since in the worst case the queue will have a size of $O(E)$, as specified above) and visiting the neighbours (which has a time complexity, in the worst case, equal to the number of edges, $O(E)$). Therefore, the final complexity is $O(E)$.

\subsection{Depth-First Search Based Algorithm}

\subsubsection{Overview}

The algorithm initializes a \texttt{visited} array to keep track of the already visited vertices and a \texttt{degree} array to store the degrees of vertices in the graph \(g\). The depth-first search (DFS) traversal begins with a vertex having the minimum degree. During the DFS, it decrements the degrees of neighbors, marking them as visited if their degrees drop below \(k\). To handle the edge case where the graph is composed of various disconnected components, the algorithm performs subsequent DFS passes on the remaining unvisited vertices until all vertices are visited. After performing all the depth-first searches, the algorithm performs a final pass over the vertices to adjust their degrees based on their neighbors with degrees greater than or equal to \(k\). Finally, vertices with degrees greater than or equal to \(k\) are collected into the \texttt{cores} list, forming the K-Cores subgraph. The code for this algorithm is presented in algorithm \ref{algo:dfs-based}.

\begin{algorithm}[htb]
    \label{algo:dfs-based}
    \caption{Depth-First Search Based Algorithm}

    \SetKwFunction{DFS}{DFS}
    \SetKwFunction{FindKCoresDFS}{findKCoresDFS}
    \SetKwProg{Fn}{Function}{:}{}
    \Fn{\DFS{$u$, $g$, $visited$, $degree$, $k$}}{
        stack $\leftarrow$ empty stack\;
        stack.push($u$)\;
        visited[$u$] $\leftarrow$ true\;

        \While{stack is not empty}{
            $v \leftarrow$ stack.pop()\;
            \If{$degree[v] < k$}{
                \ForEach{neighbor $w$ of $v$ in $g$}{
                    \If{not visited[$w$]}{
                        stack.push($w$)\;
                        visited[$w$] $\leftarrow$ true\;
                        $degree[w] \leftarrow degree[w] - 1$\;
                    }
                }
            }
            \Else{
                \ForEach{neighbor $w$ of $v$ in $g$}{
                    \If{not visited[$w$]}{
                        stack.push($w$)\;
                        visited[$w$] $\leftarrow$ true\;
                    }
                }
            }
        }
    }
    \Fn{\FindKCoresDFS{$g$, $k$}}{
        visited $\leftarrow$ array of size $|V|$ (initialized to false)\;
        degree $\leftarrow$ array of size $|V|$ (initialized with degrees of vertices in $g$)\;
        startVertex $\leftarrow$ vertex with minimum degree in $g$\;

        \DFS{startVertex, $g$, visited, degree, $k$}\;

        \ForEach{vertex $v$ in $g$}{
            \If{not visited[$v$]}{
                \DFS{$v$, $g$, visited, degree, $k$}\;
            }
        }

        \ForEach{vertex $v$ in $g$}{
            \If{$degree[v] \geq k$}{
                count $\leftarrow$ 0\;
                \ForEach{neighbor $w$ of $v$ in $g$}{
                    \If{$degree[w] \geq k$}{
                        count $\leftarrow$ count + 1\;
                    }
                }
                $degree[v] \leftarrow$ count\;
            }
        }

        cores $\leftarrow$ empty list\;
        \ForEach{vertex $v$ in $g$}{
            \If{$degree[v] \geq k$}{
                cores.append($v$)\;
            }
        }

        \KwRet cores\;
    }
\end{algorithm}

\subsubsection{Complexity Analysis}

The algorithm's space complexity depends on the storage of the \texttt{visited} and \texttt{degree} arrays. Since the arrays are of size \(|V|\), the space complexity is also \(O(|V|)\), which is linear in the number of vertices of the graph.

The algorithm's time complexity is determined by the DFS traversals which are known to have a time complexity of \(O(V + E)\). Therefore, the final time complexity is \(O(V + E)\).
    
    

\section{Evaluation}
\label{Evaluation}

In order to evaluate the performance of both alorithms, we implemented them in C++ \footnote{\url{https://github.com/jsilll/crc}} using the Boost Graph Library's (BGL) \footnote{\url{https://www.boost.org/doc/libs/1_82_0/libs/graph/doc/index.html}} graph data structure. This allowed us to easily generate random graph instances and to use the BGL's built-in functions to visualize the graphs and the k-cores. Moreover, having the both algorithms implemented in the same language and using the same graph data structure allowed us to compare them in a fair way.

To benchmark the algorithms, we used various graphs from the Stanford Network Analysis Project (SNAP) \footnote{\url{https://snap.stanford.edu/data/}}. We ran the algorithms for all the $k$ values between 1 and the maximum degree of the graph. The results are presented in figures \ref{fig:amazon}, \ref{fig:cit-patents}, \ref{fig:tech-as-skitter}, \ref{fig:web-google} and \ref{fig:wikitalk}.

All the benchmarks were run on a computer with an Intel Core i5-8250U CPU and 8GB of RAM. The operating system was Fedora 38. The compiler used was Clang 16.0.4. The compiler flags used were \texttt{-O3} and \texttt{-march=native}.

\begin{figure}[H]
    \centering
    \includegraphics[width=0.6\textwidth]{Figures/Amazon0302-1.png}
    \caption{Amazon0302 benchmark times for both algorithms}
    \label{fig:amazon}
\end{figure}

\begin{figure}[H]
    \centering
    \includegraphics[width=0.6\textwidth]{Figures/cit-Patents.png}
    \caption{cit-Patents benchmark times for both algorithms}
    \label{fig:cit-patents}
\end{figure}

\begin{figure}[H]
    \centering
    \includegraphics[width=0.6\textwidth]{Figures/tech-as-skitter.png}
    \caption{tech-as-skitter benchmark times for both algorithms}
    \label{fig:tech-as-skitter}
\end{figure}

\begin{figure}[H]
    \centering
    \includegraphics[width=0.6\textwidth]{Figures/web-Google.png}
    \caption{web-Google benchmark times for both algorithms}
    \label{fig:web-google}
\end{figure}

\begin{figure}[H]
    \centering
    \includegraphics[width=0.6\textwidth]{Figures/WikiTalk.png}
    \caption{WikiTalk benchmark times for both algorithms}
    \label{fig:wikitalk}
\end{figure} 

The results consistently demonstrate that the depth-first search-based algorithm outperforms the degree-pruning algorithm significantly. Several factors contribute to this performance difference. 

To begin with, the process of removing vertices and edges in the degree-pruning algorithm is computationally expensive. In contrast, the depth-first search-based algorithm simply maintains information about vertex degrees and visited vertices, which is a considerably simpler task. 

Secondly, the degree-pruning algorithm allows a node to be inserted multiple times, leading to a larger queue that needs to be processed. In contrast, the depth-first search-based algorithm traverses each node only once by effectively tracking visited nodes.

Lastly, the degree-pruning algorithm is akin to a breadth-first search approach tailored to uncover the k-cores of a graph. Given this resemblance and the specific benchmarks considered, opting for a depth-first search-based algorithm offers performance benefits, resulting in consistently faster execution times than the degree-pruning algorithm.

\section{Future Work}
The exploration of k-core decomposition algorithms has provided valuable insights into understanding the structure of complex networks. However, there are several avenues for future research that could enhance the field and contribute to the development of more efficient algorithms.

\subsection{Parallelization with OpenMP and MPI}

One of the key areas for future research involves investigating the parallelization of both k-core decomposition algorithms using Open Multi-Processing \footnote{\url{https://www.openmp.org/}} (OpenMP) and Message Passing Interface \footnote{\url{https://www.open-mpi.org/}} (MPI). Implementing parallel computing paradigms can significantly accelerate the execution of algorithms, especially when dealing with large-scale networks. OpenMP, a shared-memory parallelization framework, and MPI, a distributed-memory parallelization protocol, offer opportunities to exploit multi-core processors and distributed computing environments effectively. By parallelizing the algorithms, researchers can distribute the computational load across multiple processors or even across clusters of computers, thereby substantially reducing the processing time for k-core decomposition. This approach could be particularly beneficial for handling massive real-world networks encountered in various applications, such as social networks, biological networks, and communication networks.


\subsection{Exploring Performance Discrepancies}


Considering the impact of network characteristics on algorithm performance is crucial. Different types of networks, such as scale-free networks, small-world networks, or dense social networks, may pose unique challenges for k-core decomposition algorithms. Future research could focus on tailoring algorithms to specific network topologies, ensuring that the algorithms' performance remains consistent across various real-world scenarios. These efforts will not only advance the field of network analysis but also pave the way for more accurate and timely analyses of complex real-world networks.

\label{FutureWork}

\section{Conclusion}
\label{Conclusion}

The K-Cores Problem represents a powerful paradigm in graph theory and network analysis, offering nuanced insights into the intricate world of interconnected systems. By delving into cohesive substructures within networks, this concept has far-reaching implications across various domains.

In social networks, the identification of K-Cores unravels tightly-knit communities, enhancing our understanding of human interactions and social dynamics. In biological networks, it aids in pinpointing essential elements, paving the way for advancements in disease research and drug discovery. When applied to internet and web graphs, K-Cores illuminate the backbone of the digital world, spotlighting pivotal websites and content hubs. Furthermore, in transportation and infrastructure networks, K-Cores pinpoint critical nodes, optimizing traffic flow and ensuring the robustness of urban planning strategies.

The ability to discern K-Cores is not merely a theoretical endeavor; it is a practical tool for decision-makers, researchers, and analysts. By leveraging the insights derived from K-Cores, professionals can make informed decisions, design more resilient systems, and enhance the efficiency of various real-world applications.

As networks continue to evolve and expand in complexity, the K-Cores Problem remains a cornerstone in understanding their underlying structures. Its applications in diverse fields underscore its significance, making it an indispensable asset in the toolkit of network analysts and researchers worldwide.


\clearpage
\bibliographystyle{abbrv}
\bibliography{Bibliography}
\end{document}