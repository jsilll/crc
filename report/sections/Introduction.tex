\label{sec:Introduction}

In the realm of network science and graph theory, the identification of cohesive substructures within complex networks is a fundamental task with far-reaching implications. One such critical concept is the k-core of a graph, a mathematical construct that characterizes the inherent structure and resilience of a network. K-cores offer valuable insights into the core-periphery organization of networks and have applications in various fields, including social network analysis, biology, transportation systems, and more.

The k-core of a graph represents a maximal subgraph where every vertex is connected to at least k other vertices within the subgraph. This concept is essential for understanding the robustness and functionality of real-world networks. As the value of k increases, the resulting k-core becomes progressively denser, unveiling the inner layers of a network's connectivity.

The process of identifying k-cores is not a trivial one, and researchers have developed multiple algorithms to efficiently compute them. These algorithms serve as the cornerstone for uncovering the intricate hierarchical structure within networks and have significant implications for various network-related applications.

This report embarks on a comparative study of two distinct algorithms for finding k-cores within a given graph. Our primary objective is to explore and evaluate the strengths and weaknesses of these algorithms in terms of computational efficiency, accuracy, and scalability. By doing so, we aim to provide valuable insights into the selection of an appropriate algorithm based on specific network characteristics and research objectives.

The two algorithms under consideration, which will be thoroughly analyzed and compared in this report, are the \textit{Degree Pruning Algorithm} and an algorithm that uses \textit{DFS} to solve the problem. Both algorithms have gained recognition in the field of network science and have shown promise in identifying k-cores effectively. However, they employ different strategies and heuristics, which may lead to variations in their performance under different network scenarios.

In the subsequent sections of this report, we will delve into the theoretical foundations of k-cores, provide an in-depth explanation of the chosen algorithms, present our experimental methodology, and report the findings of our comparative analysis. By the conclusion of this study, readers will have a comprehensive understanding of the intricacies involved in identifying k-cores and will be equipped with insights to make informed decisions about the choice of algorithm for their specific network analysis tasks.
